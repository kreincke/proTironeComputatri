% This file is part of the Open Source project 'proTironeComputatri'
% (c) 2025 Karsten Reincke (https://github.com/kreincke/proTironeComputatri)
% It is distributed under the terms of the creative commons license
% CC-BY-4.0 (= https://creativecommons.org/licenses/by/4.0/)

\documentclass[
  DIV=calc,
  BCOR=5mm,
  11pt,
  headings=small,
  oneside,
  abstract=true,
  toc=bib,
  english,ngerman]{scrartcl}

% customize your presentation
% ---------------------------
% set paths:

\def\imgGl{../../img.gl/}
\def\bibGl{../../bib.gl}
\def\cfgGl{../../cfg.gl/}

\usepackage[utf8]{inputenc}
\usepackage{a4}
\usepackage[english,ngerman]{babel}

\usepackage[
  backend=biber,
  style=authortitle-dw,
  sortlocale=auto,
]{biblatex}
\input{\cfgGl/inc.cfg-biber-de.tex}

\addbibresource{\bibGl/lit.fachinformatik.bib}

% package for improving the grey value and the line feed handling
\usepackage{microtype}

%language specific quoting signs
\usepackage[style=german,autostyle=true,]{csquotes}

% language specific hyphenation
\input{\cfgGl/inc.babelhyphenations.tex}

%%% (3) layout page configuration %%%

% select the visible parts of a page
% S.31: { plain|empty|headings|myheadings }
\pagestyle{headings}
% select the wished style of page-numbering
% S.32: { arabic,roman,Roman,alph,Alph }
\pagenumbering{arabic}
\setcounter{page}{1}

% select the wished distances using the general setlength order:
% S.34 { baselineskip| parskip | parindent }
% - general no indent for paragraphs
\setlength{\parindent}{0pt}
\setlength{\parskip}{1.2ex plus 0.2ex minus 0.2ex}


%- start(footnote-configuration)

\deffootnote[1.5em]{1.5em}{1.5em}{\textsuperscript{\thefootnotemark)\ }}

%for using label as nameref
\usepackage{nameref}

%integrate nomenclature
\input{\cfgGl/inc.cfg-ncl-de.tex}

% Hyperlinks
\usepackage{hyperref}
\hypersetup{bookmarks=true,breaklinks=true,colorlinks=true,citecolor=blue,draft=false}

\usepackage{multirow,tabularx}
\usepackage{fontawesome}
\usepackage{rotating}
\usepackage[dvipsnames]{xcolor}

\begin{document}
\selectlanguage{ngerman}

%% use all entries of the bliography
\nocite{*}

\titlehead{Ausbildung zur Fachinformatikerin}
\subject{Release 0.1}
\title{Crossover Literaturverzeichnis}

\subtitle{gruppiert nach Lernfeldern}
\author{Karsten Reincke\input{\cfgGl/inc.lfn-dd.tex}}

\maketitle

(Hauptsächlich) nach Lernfeldern gruppiert, listen wir hier Lehr- und Lernbücher auf, die - mehr oder minder absichtlich - den Stoff für die Ausbildung zur Fachinformatikerin aufbereiten:

\printbibliography[keyword={ORGA}, title={Duale Ausbildung und Formalitäten}]
\printbibliography[keyword={LF01}, title={Lernfeld 01: Das Unternehmen und die eigenene Rolle im Betrieb}]
\printbibliography[keyword={LF02}, title={Lernfeld 02: Arbeitsplätze einrichten}]
\printbibliography[keyword={LF03}, title={Lernfeld 03: Netzwerkclients einbinden}]
\printbibliography[keyword={LF04}, title={Lernfeld 04: Schutzbedarf analysieren}]
\printbibliography[keyword={LF05}, title={Lernfeld 05: Software anpassen }]
\printbibliography[keyword={LF06}, title={Lernfeld 06: Serviceanfragen bearbeiten}]
\printbibliography[keyword={LF07}, title={Lernfeld 07: Cyber-physische Systeme ergänzen}]
\printbibliography[keyword={LF08}, title={Lernfeld 08: Daten systenübergreifend bereitstellen}]
\printbibliography[keyword={LF09}, title={Lernfeld 09: Netzwerke und Dienste bereitstellen}]
\printbibliography[keyword={LF10a}, title={FIAE/Lernfeld 10a: Benutzerschnittstellen gestalten}]
\printbibliography[keyword={LF11a}, title={FIAE/Lernfeld 11a: Funktionalität in Anwendungen realisieren}]
\printbibliography[keyword={LF12a}, title={FIAE/Lernfeld 12a: Anwendungen kundenspezifisch entwickln}]
\printbibliography[keyword={LF10b}, title={FISI/Lernfeld 10b: Serverdienste bereitstellen und Administration automatisieren}]
\printbibliography[keyword={LF11b}, title={FISI/Lernfeld 11b: Betrieb und Sicherheit vernetzter Systeme gewährleisten}]
\printbibliography[keyword={LF12b}, title={FISI/Lernfeld 12b: Systeme kundenspezifisch integrieren}]
\printbibliography[keyword={LF10c}, title={FIDP/Lernfeld 10c: Werkzeuge maschinellen Lernens einsetzen}]
\printbibliography[keyword={LF11c}, title={FIDP/Lernfeld 11c: Prozesse analysieren und gestalten}]
\printbibliography[keyword={LF12c}, title={FIDP/Lernfeld 12c: Prozesse und Daten kundenspezifisch analysieren}]
\printbibliography[keyword={LF10d}, title={FIDV/Lernfeld 10d: Cyber-physische Systeme entwickeln}]
\printbibliography[keyword={LF11d}, title={FIDV/Lernfeld 11d: Betrieb und Sicherheit vernetzter Systeme gewährleisten}]
\printbibliography[keyword={LF12d}, title={FIDV/Lernfeld 12d: Cyber-physische Systeme kundenspezifisch optimieren}]
\printbibliography[keyword={SLBM}, title={Zeit- und Selbstmanagement}]
\printbibliography[notkeyword={ORGA}, 
  notkeyword={LF01},  notkeyword={LF02}, notkeyword={LF03},  notkeyword={LF04}, notkeyword={LF05},
  notkeyword={LF06},  notkeyword={LF07},  notkeyword={LF08},  notkeyword={LF09},  
  notkeyword={LF10a}, notkeyword={LF11a}, notkeyword={LF12a},
  notkeyword={LF10b}, notkeyword={LF11b}, notkeyword={LF12b},
  notkeyword={LF10c}, notkeyword={LF11c}, notkeyword={LF12c},
  notkeyword={LF10d}, notkeyword={LF11d}, notkeyword={LF12d},
notkeyword={SLBM}, title={Sonstiges}]


\input{\bibGl/ncl.abbrevs-de.tex}
\input{\bibGl/ncl.journals.tex}
\printnomenclature

\end{document}
